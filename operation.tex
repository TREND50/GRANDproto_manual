\section{Operation} 

Here we describe how to start and run the GRANDproto35 setup.

\subsection{Start up}
\label{startup}
\begin{enumerate}[-]
\item{ First set-up, connect and power the Front-End Unit as described in section \ref{feu}.} 
%\item{Then make sure that the environment variable \texttt{\$DATADIR} is properly defined through the command \texttt{echo \$DATADIR}. This variable defines where the data should be stored. It should be defined in one of the config file ( \texttt{.profile} for instance).}
%
\item{Issue the following command should be launched on the DAQ PC: \\
\ \\
\texttt{setAdress.sh BOARDID} \\
\ \\
\hl{Note here that BOARDID is 2 digits for ALL units (ie 01 for n$^{\circ}$1)}. \\
This command will call the subprocess \texttt{configureGedek} located in subfolder \texttt{setAdress/}. This program, written in C, will attribute a dedicated IP adress 192.168.1.1xx to each board of ID number xx. Communication of \texttt{configureGedek} with the board is done through its MAC adresss stored in the dedicated file \texttt{configGedek.txt}. \\
Success in establishing the proper IP adress to a unit can be checked through a ping command to the unit IP adress. Note that the \texttt{setAdress.sh} script also informs the units about the DAQ PC ID (IP adress and MAC adress). If these 2 parameters do not correspond to the actual values of the PC, then DAQ communication will not be possible (see troubleshooting section).}
%
\item{Then the status of the unit can be requested, through the following command: \\
\ \\
\texttt{slcreq.sh  BOARDID }\\
\ \\
A slow-control event file should appear in the \texttt{\$DATADIR} folder. Edit it and check that all voltage values are correct (Voltage1=14V, Voltage2=3V, Voltage3=4V, Voltage4 to 6= 12V) and that temeprature is also reasonable (between 10 and 50$^{\circ}$C).  If all is OK at this stage, this means that DAQ communication with teh board works fine and that the Unit is properly powered.}
%
\item{Then launch:\\
\ \\
\texttt{pattern.sh BOARDID 3} \\
\ \\
This executes a {\it pattern} run in toggle mode (see section \ref{pattern}) and allows to check that the logical section of the board is running properly.}
%
\item{Then launch: \\
\ \\
\texttt{loopCalib.sh BOARDID 60 60} \\
\ \\
This execute a calibration run with a total attenuation of {\bf xx} dB. Open the run file with python script with \texttt{anaData.py} and check that the average value of the signal is around a value of {\bf xx} Volts, with a standard deviation not above {\bf xx} Volts. This allows to check that analog section of the board is running properly. }
%
\item{Then launch: \\
\ \\
\texttt{minBias.sh BOARDID} \\
\ \\
Open the run file with python script with \texttt{anaData.py} and check that the average value of the signal is around a value of {\bf xx} Volts, with a standard deviation not above {\bf xx} Volts. This allows to check the status of the antenna and of its electromagnetic environment.Check also in the event header that the GPS info is present.}
%
\item{Finaly launch: \\
\ \\
\texttt{phys.sh BOARDID 60 60 60 60 1000 10000} \\
\ \\
This will start a physics run, with triggers on the X and Y channels. Monitor the trigger rate by launching \texttt{loopSLC.sh}, which will request slow control data every 10 seconds. Display the trigger rate with the \texttt{anaSLC.py} script.}
\end{enumerate} 

\subsection{Normal operation}

\subsection{Monitoring data}

\subsection{Calibration}
